\PassOptionsToPackage{unicode=true}{hyperref} % options for packages loaded elsewhere
\PassOptionsToPackage{hyphens}{url}
%
\documentclass[]{article}
\usepackage{lmodern}
\usepackage{amssymb,amsmath}
\usepackage{ifxetex,ifluatex}
\usepackage{fixltx2e} % provides \textsubscript
\ifnum 0\ifxetex 1\fi\ifluatex 1\fi=0 % if pdftex
  \usepackage[T1]{fontenc}
  \usepackage[utf8]{inputenc}
  \usepackage{textcomp} % provides euro and other symbols
\else % if luatex or xelatex
  \usepackage{unicode-math}
  \defaultfontfeatures{Ligatures=TeX,Scale=MatchLowercase}
\fi
% use upquote if available, for straight quotes in verbatim environments
\IfFileExists{upquote.sty}{\usepackage{upquote}}{}
% use microtype if available
\IfFileExists{microtype.sty}{%
\usepackage[]{microtype}
\UseMicrotypeSet[protrusion]{basicmath} % disable protrusion for tt fonts
}{}
\IfFileExists{parskip.sty}{%
\usepackage{parskip}
}{% else
\setlength{\parindent}{0pt}
\setlength{\parskip}{6pt plus 2pt minus 1pt}
}
\usepackage{hyperref}
\hypersetup{
            pdftitle={Rapport du travail effectu sur les TP1 et 2 d'algorithmique gnrale},
            pdfauthor={Arthur Zucker},
            pdfborder={0 0 0},
            breaklinks=true}
\urlstyle{same}  % don't use monospace font for urls
\usepackage{graphicx,grffile}
\makeatletter
\def\maxwidth{\ifdim\Gin@nat@width>\linewidth\linewidth\else\Gin@nat@width\fi}
\def\maxheight{\ifdim\Gin@nat@height>\textheight\textheight\else\Gin@nat@height\fi}
\makeatother
% Scale images if necessary, so that they will not overflow the page
% margins by default, and it is still possible to overwrite the defaults
% using explicit options in \includegraphics[width, height, ...]{}
\setkeys{Gin}{width=\maxwidth,height=\maxheight,keepaspectratio}
\setlength{\emergencystretch}{3em}  % prevent overfull lines
\providecommand{\tightlist}{%
  \setlength{\itemsep}{0pt}\setlength{\parskip}{0pt}}
\setcounter{secnumdepth}{0}
% Redefines (sub)paragraphs to behave more like sections
\ifx\paragraph\undefined\else
\let\oldparagraph\paragraph
\renewcommand{\paragraph}[1]{\oldparagraph{#1}\mbox{}}
\fi
\ifx\subparagraph\undefined\else
\let\oldsubparagraph\subparagraph
\renewcommand{\subparagraph}[1]{\oldsubparagraph{#1}\mbox{}}
\fi

% set default figure placement to htbp
\makeatletter
\def\fps@figure{htbp}
\makeatother


\title{Rapport du travail effectu sur les TP1 et 2 d'algorithmique gnrale}
\author{Arthur Zucker}
\date{08/02/2019}

\begin{document}
\maketitle

\texttt{\{r\ setup,\ include=FALSE\}\ knitr::opts\_chunk\$set(echo\ =\ TRUE)}
\# Rapport du travail effectué sur les TP1 et 2 d'algorithmique générale
Inline \texttt{code} has \texttt{back-ticks\ around} it.

\begin{verbatim}
pwd
ls
\end{verbatim}

\hypertarget{le-parcour-dfs-va-pour-ma-structure-renomer-les-points-en-111-pour-1.1.1}{%
\section{Le parcour DFS va pour ma structure renomer les points en 111
pour
1.1.1}\label{le-parcour-dfs-va-pour-ma-structure-renomer-les-points-en-111-pour-1.1.1}}

Ce parcour va au commencer en 1 point aléatoire, pour renvoyer une
matrice contenant les sommets par lesquels il est passé durant son
parcour, et va ensuite effectuer l'algorythm pour d'autre sommet A
\textbar{} **** \textbar{} A \textbar{}c \textbar{}s \# La connexité se
vérifie en while(incrémenter i != 0) On teste tous les chemins de tous
les poinsts

\hypertarget{pour-la-partie-b}{%
\section{Pour la partie B :}\label{pour-la-partie-b}}

Algorithme de Kosaraju!!!! \includegraphics{./Images/sorbonne.png}

\end{document}

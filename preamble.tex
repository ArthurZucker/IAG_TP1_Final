\usepackage{moreverb}								% List settings
%\usepackage{textcomp}								% Fonts, symbols etc.
								% Latin modern font
\usepackage[T1]{fontenc}
\usepackage{palatino}
\usepackage[francais]{babel}							% Language settings
\usepackage[utf8]{inputenc}							% Input settings
\usepackage{amsmath}								% Mathematical expressions (American mathematical society)
\usepackage{amssymb}								% Mathematical symbols (American mathematical society)
\usepackage{graphicx}								% Figures
\usepackage{subfig}									% Enables subfigures
\numberwithin{equation}{chapter}					% Numbering order for equations
\numberwithin{figure}{chapter}						% Numbering order for figures
\numberwithin{table}{chapter}						% Numbering order for tables
\usepackage{listings}								% Enables source code listings
\usepackage[top=2.9cm, bottom=2.8cm,
			inner=2.5cm, outer=2.5cm]{geometry}
\usepackage{parskip}								% Enables vertical spaces correctly
\usepackage[dvipsnames]{xcolor}
\definecolor{BleuSorbonne}{RGB}{46,41,90}
\definecolor{BleuPolytech}{RGB}{0,157,223}

% OPTIONAL SETTINGS (DELETE OR COMMENT TO SUPRESS)

% Caption settings (aligned left with bold name)

\usepackage[pdfpagelabels]{hyperref}

% Define the number of section levels to be included in the t.o.c. and numbered	(3 is default)
\setcounter{tocdepth}{5}
\setcounter{secnumdepth}{5}


% Chapter title settings
\usepackage{titlesec}
\titleformat{\chapter}[display]
  {\Huge\bfseries\filcenter\sffamily\color{BleuSorbonne}}
  {{\fontsize{40pt}{1em}\vspace{-3.9ex}\selectfont \sffamily{\thechapter}}}{1ex}{}[]
\titleformat{\section}[hang]
  {\Large\sffamily\color{BleuPolytech}}
  {{\fontsize{18pt}{}\selectfont \sffamily{\thesection}}}{1ex}{}[]

% Header and footer settings (Select TWOSIDE or ONESIDE layout below)
\usepackage{fancyhdr}
\pagestyle{fancy}

\renewcommand{\chaptermark}[1]{\markboth{\thechapter.\space#1}{}}
\fancyhf{}
\addtolength{\footskip}{28pt}
\renewcommand{\footrulewidth}{0.4pt}
\fancyhead[C]{\nouppercase{ \leftmark}}
\fancyfoot[C]{\thepage}
\fancyfoot[L]{\includegraphics[width=2.7cm]{media/SU_SCIENCES.png}}	\fancyfoot[R]{\includegraphics[width=3.5cm]{media/polytech_sorbonne.jpg}}

\fancypagestyle{plain}{			% Redefine the plain page style
\fancyhf{}

\renewcommand{\headrulewidth}{0pt}
\renewcommand{\footrulewidth}{0.4pt}
\fancyfoot[C]{\thepage}
\fancyfoot[L]{\includegraphics[width=2.7cm]{media/SU_SCIENCES.png}}	\fancyfoot[R]{\includegraphics[width=3.5cm]{media/polytech_sorbonne.jpg}}}

\usepackage[autolanguage]{numprint}
\setlength{\parindent}{25pt}

\newpage
\thispagestyle{empty}
\begin{figure}[h!]
	\includegraphics[width=0.28\pdfpagewidth]{media/SU_SCIENCES.png}
	\hfill
	\includegraphics[width=0.37\pdfpagewidth]{media/polytech_sorbonne.jpg}
\end{figure}

\vspace{5\baselineskip}

\begin{center}
    \Large
    Algorithmique\\
    \vspace{2\baselineskip}
	\textbf{\Huge\color{bleuperso} Rapport du TP1}\\
	\vspace{2\baselineskip}
	réalisé par\\
    \vspace{2\baselineskip}
	Thomas Genin\\
	\vfill
	\textsc{Polytech Sorbonne}\\
    \vspace{0.5\baselineskip}
	Spécialité\\
    \vspace{0.5\baselineskip}
	Mathématiques Appliquées et Informatique Numérique\\
    \vspace{0.5\baselineskip}
	3\up{ième} année\\
    \vspace{0.5\baselineskip}
    \Large Paris, France \the\year
\end{center}
